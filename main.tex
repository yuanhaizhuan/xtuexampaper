% !TeX encoding = UTF-8
% !TeX program = xelatex
% !TeX spellcheck = en_US
%%%%%%%%%%%%%%%%%%%%%%%%%%%%%%%%%%%%%%%%%
%
% 湘潭大学考试试卷模板
% Haizhuan Yuan (袁海专)
% 版本 1.0 (22/12/2020)
%
%%%%%%%%%%%%%%%%%%%%%%%%%%%%%%%%%%%%%%%%%

\documentclass{xtuexampaper}

\begin{document}
\showanswer{1}              % 是否显示答案,0为不显示,1为显示
\school{数学与计算科学学院}    % 学院名称
\papermaker{袁海专}          % 制卷人
\reviewer{袁海专}            % 审核人
\approvaldate{\today}       % 审核日期
\copies{90}                 % 试卷份数
\serialno{}                 % 编号
\schoolyear{2020}           % 学年
\schoolterm{下}             % 学期
\studentgrade{2018}         % 年级  
\course{《试卷模板》}          % 课程名称
\controlno{A}               % A or B 卷
\major{数据科学与大数据技术}    % 适用专业
\exammethod{闭卷}            % 考试方式
\examtime{120}              % 考试时间

\PaperHeader                % 生成卷头
\ScoringSheet{6}            % 生成记分栏 参数:大题数[2-10] 

\setlength{\baselineskip}{16pt}  % 行间距设置

%%===========================================================================
%% 填空题
%% 命令:\newsubsection{#1}
%% 参数: #1 为小题题目
%% 命令: \fill{#1}{#2} 
%% 参数: #1 为填空宽度
%%       #2 为填空答案
%%===========================================================================
\newsection{填空题(每小题3分,共18分)}
%------------------------------
\newsubsection{湘潭大学数学与计算科学学院现有
	\fill{3.5cm}{数学与应用数学}、
	\fill{3.5cm}{信息与计算科学}、
	\fill{3.cm}{统计学}和
	\fill{3.5cm}{数据科学与大数据技术}
	四个本科专业。
}
%------------------------------
\newsubsection{湘潭大学数学与计算科学学院现有
	\fill{3.5cm}{数学与应用数学系}、
	\fill{3.5cm}{信息与计算科学系}、
	\fill{3.cm}{统计学系}和
	\fill{3.5cm}{高等数学教学部}等4个系部。}

%%===========================================================================
%% 选择题
%% 命令: \newchoice{#1}{#2}{#3}{#4}{#5}{#6}{#7} 
%% 参数: #1 为题目
%%       #2 为答案
%%       #3 为四选项的摆放行数,取值为 1 或 2 或 4
%%       #4-#7 依次为四个选项的内容 
%%===========================================================================
\newsection{选择题(每小题3分,共18分)}
%------------------------------
\newchoice{湘潭大学数学与计算科学学院成立于}{D}{1}
{1974年}{1976年}{1981年}{2003年}
%------------------------------
\newchoice{湘潭大学数学与计算科学学院成立于}{D}{2}
{1974年}{1976年}{1981年}{2003年}
%------------------------------
\newchoice{湘潭大学数学与计算科学学院成立于}{D}{4}
{1974年}{1976年}{1981年}{2003年}
%------------------------------
\newchoice{湘潭大学数学与计算科学学院成立于}{D}{1}
{1974年}{1976年}{1981年}{2003年}


%%===========================================================================
%% 判断题
%% 命令: \newjudgment{#1}{#2}
%% 参数: #1 为题目
%%       #2 为答案
%%===========================================================================
\newsection{判断题(每小题2分,共10分)}
\newjudgment{湘潭大学数学与计算科学学院成立于1974年。}{0}
\newjudgment{湘潭大学数学与计算科学学院成立于1976年。}{0}
\newjudgment{湘潭大学数学与计算科学学院成立于1981年。}{0}
\newjudgment{湘潭大学数学与计算科学学院成立于2003年。}{1}

%%===========================================================================
%% 计算题
%% 命令:\newsubsection{#1}
%% 参数: #1 为小题题目
%% 命令: \answer{#1}{#2}
%% 参数: #1 为答案所占页面高度
%%       #2 为答案
%%===========================================================================
\newsection{计算题(每小题7分,共14分)}
\begin{multicols}{2} %%双栏环境
%------------------------------
\newsubsection{请计算 $1 + 1$。}
\answer{3cm}{1+1 = 2}
%------------------------------
\newsubsection{请计算 $2 + 3$。}
\answer{3cm}{2+3 = 5}
\end{multicols}


%%===========================================================================
%% 简答题
%% 命令:\newsubsection{#1}
%% 参数: #1 为题目
%% 命令: \answer{#1}{#2}
%% 参数: #1 为答案所占页面高度
%%       #2 为答案
%%===========================================================================
\newsection{简答题:湘潭大学数学与计算科学学院简介。(14分)}
\answer{4cm}{
数学与计算科学学院是湘潭大学成立最早的院系之一,肇始于1974年复校之初的数学、计算数学专业。1976年成立数理系,1981年更名为数学系,2003年成立数学与计算科学学院。现有数学与应用数学系、信息与计算科学系、统计学系和高等数学教学部等4个系部。学院是全国教育系统先进集体、湖南省首批高校党建工作标杆院系。
}

%%===========================================================================
%% 程序题
%% 命令:\newsubsection{#1}
%% 参数: #1 为题目
%% 命令: \answer{#1}{#2}
%% 参数: #1 为答案所占页面高度
%%       #2 为答案
%%===========================================================================
\newsection{程序题(每小题10分,共20分)}
%------------------------------
\newsubsection{ 数据集cj描述了一些学生的部分课程成绩, 共有20条数据, 11个字段, 分别为学号, 班级, 姓名, 性别, 英语, 体育, 军训, 数分, 高代, 解几(解析几何), 总分数, 数据集已导入, 保存在DataFrame对象grade中。
}
\answer{4cm}{
	
}
%------------------------------
\newsubsection{ 数据集cj描述了一些学生的部分课程成绩, 共有20条数据, 11个字段, 分别为学号, 班级, 姓名, 性别, 英语, 体育, 军训, 数分, 高代, 解几(解析几何), 总分数, 数据集已导入, 保存在DataFrame对象grade中。
}
\answer{4cm}{
}

\end{document}
